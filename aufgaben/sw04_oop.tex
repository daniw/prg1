\section{SW4}
\subsection{Selfstudy-Questions OOP4}

\subsubsection{Chapter 3.6 - Class diagrams vs. object diagrams}

\subsubsection*{Exercise 1}
\textit{How do you declare a referencevariable?}\\
\verb?type name = initial_value;?

\subsubsection*{Exercise 2}
\textit{Draw the object diagram to the BlueJ project "'house"' from 
chapter 1.}\\
Drawed on paper. 

\subsubsection*{Exercise 3}
\textit{Draw the class diagram to the BlueJ project "'house"' from
chapter 1.}\\
Drawed on paper. 

\subsubsection*{Exercise 4}
\textit{Solve the exercises 3.1 to 3.4}\\
see \ref{book_oop4}. 

\subsubsection{Chapter 3.8 - The ClockDisplay source code}

\subsubsection*{Exercise 1}
\textit{Solve the exercise 3.5}\\
see \ref{book_oop4}. 

\subsubsection*{Exercise 2}
\textit{What is the result of the following expressions?}\\

\begin{table}[h!]
	\begin{tabular}{l c l l}
	\lstinline!(3>2)! & \lstinline!^! & \lstinline!(4>5)! & true \\
	\lstinline!(3<2)! & \lstinline!^! & \lstinline!(4>5)! & false \\
	\lstinline!(3<2)! & \lstinline!&&! & \lstinline!(4>5)! & false \\
	\lstinline!(3>2)! & \lstinline!||! & \lstinline!(4>5)! & true \\
	\lstinline?!(3>2)? &  &  & false
	\end{tabular}
\end{table}

\subsubsection*{Exercise 3}
\textit{Solve the exercises 3.6 to 3.8}\\
see \ref{book_oop4}. 

\subsubsection*{Exercise 4}
\textit{Solve the exercises 3.15 to 3.17 and 3.19}\\
see \ref{book_oop4}. 

\subsubsection*{Exercise 5}
\textit{Solve the exercise 3.21}\\
see \ref{book_oop4}. 

\subsubsection{Chapter 3.9 - Objects creating objects}

\subsubsection*{Exercise 1}
\textit{Solve the exercise 3.23}\\
see \ref{book_oop4}. 

\subsubsection{Chapter 3.10 - Multiple constructors}

\subsubsection*{Exercise 1}
\textit{Create the singatures for all possible constructors which accord
with the following object-creation.}\\
\lstinline{new Student("Peter", 34);}
\begin{lstlisting}
Student(String name, int age)
Student(String name, long age)
Student(String name, float age)
Student(String name, double age)
\end{lstlisting}

\subsubsection*{Exercise 2}
\textit{Solve the exercises 3.28 and 3.29}\\
see \ref{book_oop4}. 

\subsubsection{Chapter 3.11 - Method calls}

\subsubsection*{Exercise 1}
\textit{Solve the exercise 3.30}\\
see \ref{book_oop4}. 

\subsubsection{Chapter 3.12 - Another example of object interaction}

\subsubsection*{Exercise 1}
\textit{Solve the exercises 3.33 and 3.34}\\
see \ref{book_oop4}. 

\subsubsection{Chapter 3.13 - Using a debugger}

\subsubsection*{Exercise 1}
\textit{Solve the exercises 3.35 to 3.42}\\
see \ref{book_oop4}. 

\subsection{Exercises from book (chapter 3)}
\label{book_oop4}

\subsubsection*{Exercise 3.1}
Drawed on paper. 

\subsubsection*{Exercise 3.2}
A class diagram can only change, when the source code has changed. 

\subsubsection*{Exercise 3.3}
A object diagram can change dynamically during program execution. 

\subsubsection*{Exercise 3.4}
\begin{lstlisting}
private Instructor tutor
\end{lstlisting}

\subsubsection*{Exercise 3.5}


\subsubsection*{Exercise 3.6}
If replacementValue is negative or higher than the limit, value remains with 
the old value. \\
To inform the user about the mistake, a error message could be printed. 

\subsubsection*{Exercise 3.7}
When \lstinline{setValue()} is called with 0 as argument, value would not be 
set to 0. 

\subsubsection*{Exercise 3.8}
If limit is greater or equal to 0, value is replaced with replacemantValue in 
any case. Otherwise value stay the same independent of the value of 
replacementValue. 

\subsubsection*{Exercise 3.15}
The modulo operator returns the ??? of a division. 

\subsubsection*{Exercise 3.16}
2

\subsubsection*{Exercise 3.17}
with negative numbers the modulo is negative. 

\subsubsection*{Exercise 3.19}
$-(m-1) ... m-1$

\subsubsection*{Exercise 3.21}
\begin{lstlisting}
public void increment() {
    if (value < limit - 1) {
        value++;
    }
    else {
        value = 0;
    }
}
\end{lstlisting}

\subsubsection*{Exercise 3.23}
In the constructor of NumberDisplay value is initialiced with 0. So the time is 
0:00. 

\subsubsection*{Exercise 3.28}
It initialices the clock and set the time to the time given by the arguments. 

\subsubsection*{Exercise 3.29}
In the second constructor updateDisplay is not called because it is called in 
the method setTime. 

\subsubsection*{Exercise 3.30}
\begin{lstlisting}
p1.print(foo.pdf, false);
p1.print(bar.ps, true);
System.out.println(p1.getStatus(delaytime));
status = p1.getStatus(1);
\end{lstlisting}

\subsubsection*{Exercise 3.33}


\subsubsection*{Exercise 3.34}
Drawed on paper

\subsubsection*{Exercise 3.35}


\subsubsection*{Exercise 3.36}


\subsubsection*{Exercise 3.37}


\subsubsection*{Exercise 3.38}


\subsubsection*{Exercise 3.39}


\subsubsection*{Exercise 3.40}


\subsubsection*{Exercise 3.41}


\subsubsection*{Exercise 3.42}

\subsection{Learnteam Questions OOP4}

\subsubsection*{Exercise 1}


\subsubsection*{Exercise 2}
\subsubsection*{3.9}
\begin{lstlisting}
! (4 < 5) = false
! false = true
(2 > 2) || ((4 == 4) && (1 < 0)) = false
(2 > 2) || (4 == 4) && (1 < 0) = false
(34 != 33) && ! false = true
\end{lstlisting}
\subsubsection*{3.10}
\begin{lstlisting}
a == b
\end{lstlisting}
\subsubsection*{3.11}
\begin{lstlisting}
a != b
a ^ b
\end{lstlisting}
\subsubsection*{3.12}
\begin{lstlisting}
!(!a || !b)
\end{lstlisting}
\subsubsection*{3.13}
getDisplayValue does only work with two digits. therefore limit must be 
between 10 and 99. 
\subsubsection*{3.14}
there is no difference
\subsubsection*{3.18}
0, 1, 2, 3, 4
\subsubsection*{3.19}
-(m-1) ... m-1
\subsubsection*{3.20}
value is incremented by 1. if it reaches limit value is set to 0. 
\subsubsection*{3.24}
60 times \\
disp.setTime(1, 0)
\subsubsection*{3.26}
\begin{lstlisting}
Editor(String filename, int number)
\end{lstlisting}
\subsubsection*{3.27}
\begin{lstlisting}
Rectangle window = new Rectangle(100, 200);
\end{lstlisting}

\subsubsection*{Exercise 3}
\subsubsection*{3.31}
\lstinputlisting{clock-display1/ClockDisplay.java}
\lstinputlisting{clock-display1/NumberDisplay.java}
\lstinputlisting{clock-display2/ClockDisplay.java}
\lstinputlisting{clock-display2/NumberDisplay.java}
\subsubsection*{3.32}

\subsubsection*{Exercise 4}

