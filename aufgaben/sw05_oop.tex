\section{SW4}
\subsection{Selfstudy-Questions OOP5}

\subsubsection{Chapter 4.1 to 4.3 - An organizer for music files}

\subsubsection*{Exercise 1}
\textit{Solve the exercises 4.1 to 4.3}\\

\subsubsection*{Exercise 2}
\textit{What do you understand by "Java-Package"?}\\
A library built in in java. 

\subsubsection*{Exercise 3}
\textit{You want to use the library-class ArrayList. What expression makes it
possible to use that library-class in your source code?}\\
\begin{lstlisting}
import java.util.ArrayList;
\end{lstlisting}

\subsubsection{Chapter 4.4 to 4.7 - Numbering within collections}

\subsubsection*{Exercise 4}
\textit{Solve the exercises 4.4 to 4.7}\\

\subsubsection*{Exercise 5}
\textit{Solve the exercises 4.8 to 4.11}\\

\subsubsection*{Exercise 6}
\textit{Solve the exercises 4.12 to 4.13}\\

\subsubsection*{Exercise 7}
\textit{Explanin the following declaration:}\\
\lstinline?private ArrayList<Balloon> list = new ArrayList<>();?\\
A field called list is defined as a collection of Balloons

\subsubsection*{Exercise 8}
\textit{What is the connection between abstraction and ArrayLists?}\\
We do not have to know how ArrayLists work internally. It is enough to know 
how to work with it. 

\subsubsection*{Exercise 9}
\textit{What is the difference of the methods remove() and get() on
ArrayLists?}\\
remove() deletes a item in a collection whereas get returns its content. 

\subsubsection{Chapter 4.8 to 4.12 - The Iterator type}

\subsubsection*{Exercise 10}
\textit{Solve the exercises 4.18 to 4.19}\\

\subsubsection*{Exercise 11}
\textit{Solve the exercise 4.22}\\

\subsubsection*{Exercise 12}
\textit{Explain as detailed as possible the source code on page 108.}\\


\subsubsection*{Exercise 13}
\textit{Is it possible, that the body of an while-loop is never executed?}\\
Yes, if the condition is false in the first loop. 

\subsubsection*{Exercise 14}
\textit{Show two alternative expressions for no++}\\
\begin{lstlisting}
no += 1;
no = no + 1;
\end{lstlisting}

\subsubsection*{Exercise 15}
\textit{An ArrayList can be traversed by an foreach-loop. Do you know other
ways to do the same?}\\

\subsubsection*{Exercise 16}
\textit{Is hasNext() a method of ArrayList or Iterator? How do you have to 
understand/interpret the return-value of hasNext()?}\\

\subsubsection{Chapter 4.14 - Summary of the music-organizer project}

\subsubsection*{Exercise 17}
\textit{DO NOT READ THIS CHAPTER, JUST READ THE CONCEPT-BOX AT PAGE 130.}\\

\subsubsection*{Exercise 18}
\textit{A variable that is declared for a classtype (or so called 
reference-variable) can store the special value null. Explain the situation 
with a drwing/sektch. What does it look like, if it's storing an object?}\\

\subsubsection{Chapter 4.15 to 4.17 - Summary}

\subsubsection*{Exercise 19}
\textit{Solve the exercises 4.62 to 4.65}\\

\subsubsection*{Exercise 20}
\textit{Solve the exercises 4.66 to 4.68}\\

\subsubsection*{Exercise 21}
\textit{What are the pros and cons of Arrays?}\\
The access to arrays is faster than to collections. \\
Arrays can store primitive types. \\
The length of an array cannot be changed. 

\subsubsection*{Exercise 22}
\textit{How do you get the length of an Array?}\\
\begin{lstlisting}
array.length
\end{lstlisting}

\subsubsection*{Exercise 23}
\textit{Solve the exercises 4.69, 4.71, 4.73 and 4.74}\\

\subsubsection{Questions from the book}

\subsubsection{Exercise 4.1}

\subsubsection{Exercise 4.2}
I expect no error because files.remove(index); is not executed in this case 
because files.size is 0. So the if statement os false. 

\subsubsection{Exercise 4.3}
The second file is returned because the files are shifted when one is deleted. 

\subsubsection{Exercise 4.4}
\begin{lstlisting}
private ArrayList<Book> library;
\end{lstlisting}

\subsubsection{Exercise 4.5}
\begin{lstlisting}
ArrayList<Student> cs101
\end{lstlisting}

\subsubsection{Exercise 4.6}
\begin{lstlisting}
private ArrayList<MusicTrack> tracks
\end{lstlisting}

\subsubsection{Exercise 4.7}
\begin{lstlisting}[caption=without diamond notation]
library = new ArrayList<Book>();
cs101 = new ArrayList<Student>();
track = new ArrayList<MusicTrack>();
\end{lstlisting}
\begin{lstlisting}[caption=with diamond notation]
library = new ArrayList<>();
cs101 = new ArrayList<>();
track = new ArrayList<>();
\end{lstlisting}

\subsubsection{Exercise 4.8}
10

\subsubsection{Exercise 4.9}
\begin{lstlisting}
System.out.println(items.get());
\end{lstlisting}

\subsubsection{Exercise 4.10}
14

\subsubsection{Exercise 4.11}
\begin{lstlisting}
files.add(favouriteTrack)
\end{lstlisting}

\subsubsection{Exercise 4.12}
\begin{lstlisting}
dates.remove(2)
\end{lstlisting}

\subsubsection{Exercise 4.13}
5

\subsubsection{Exercise 4.18}
\begin{lstlisting}
public String listAllFiles() {
    
}
\end{lstlisting}

\subsubsection{Exercise 4.19}
\begin{lstlisting}
public String listAllFiles() {
    String returnString = "";
    for(String filename : files) {
        returnString += filename + "\n";
    }
    return(returnString)
}
\end{lstlisting}

\subsubsection{Exercise 4.22}


\subsubsection{Exercise 4.62}


\subsubsection{Exercise 4.63}


\subsubsection{Exercise 4.64}


\subsubsection{Exercise 4.65}


\subsubsection{Exercise 4.66}


\subsubsection{Exercise 4.67}


\subsubsection{Exercise 4.68}


\subsubsection{Exercise 4.69}


\subsubsection{Exercise 4.71}


\subsubsection{Exercise 4.73}


\subsubsection{Exercise 4.74}

