% coding:utf-8

%PRG1, a LaTeX-Code for a the solution of the exercises from "programming and algorithms"
%Copyright (C) 2013, Daniel Winz

%This program is free software; you can redistribute it and/or
%modify it under the terms of the GNU General Public License
%as published by the Free Software Foundation; either version 2
%of the License, or (at your option) any later version.

%This program is distributed in the hope that it will be useful,
%but WITHOUT ANY WARRANTY; without even the implied warranty of
%MERCHANTABILITY or FITNESS FOR A PARTICULAR PURPOSE.  See the
%GNU General Public License for more details.
%----------------------------------------

\section{SW1}

\subsection{Auftrag Selbststudium}

\begin{enumerate}
\subsubsection{Kapitel 1.2}
\item Kreis und Quadrat generieren \\
      Rechtsklick auf Klasse Circle $\rightarrow$ New Circle() $\rightarrow$ bestätigen\\
      Rechtsklick auf Klasse Square $\rightarrow$ New Square() $\rightarrow$ bestätigen
\item Beim Erstellen kann ein Name angegeben werden. 

\subsubsection{Kapitel 1.3}
\item Verschiebung um 40 bzw. 60 Pixel nach unten \\
      \verb?makeInvisible()?: Es bleibt sichtbar
\item 

\subsubsection{Kapitel 1.4}
\item \verb?moveHorizontal(-70)?
\item Um Methoden Werte mitzugeben
\item Der Datentyp muss der selbe sein, wie in der Methode definiert
\item Ja, lesbarkeit wird besser

\subsubsection{Kapitel 1.5}
\item Bei unzulässiger Farbe wird die Farbe "'schwarz"' gesetzt\\
      Fehlermeldung "'Error: cannot find symbol"'
\item Ohne "'"' wird nach einer Variable mit dem Namen gesucht

\subsubsection{Kapitel 1.6}
\item $~$

\subsubsection{Kapitel 1.7}
\item $~$

\subsubsection{Kapitel 1.8}
\item Haus aus Quadrat und Dreieck, Sonne aus Kreis \\
      move, change size, change color...
\item In Feldern abgespeicherte Variabeln
\item Zustand - aktueller Zustand des Objekts

\subsubsection{Kapitel 1.9}
\item $~$
\item $~$

\subsubsection{Kapitel 1.10}
\item Eine Klasse beschreibt die Eigenschaften der konkreten Objekte
\item Grünes Dreieck, rotes Quadrat, schwarzes Quadrat, gelber Kreis, move, 
      change size, change color...

\subsubsection{Kapitel 1.11}
\item $~$
\item \begin{lstlisting}[caption=Aufgabe 1.18]
        ...
        sun = new Circle();
        sun.changeColor("yellow");
        sun.moveHorizontal(100);
        sun.moveVertical(-40);
        sun.changeSize(80);
        sun.makeVisible();
        sun.slowMoveVertical(120);
    }
\end{lstlisting}
\item 
\begin{lstlisting}[caption=Aufgabe 1.19]
/**
* Sunset
*/
public void sunset()
{
    sun.slowMoveVertical();
}
\end{lstlisting}

\subsubsection{Kapitel 1.12 und 1.13}
\item $~$

\subsubsection{Kapitel 1.14}
\item $~$
\item $~$
\item Um Rückgabewerte von Methoden weiterzuverwenden
\item void

\subsubsection{1.15}
\item 
1.31: \\
\begin{itemize}\item int \item string \item int \item int \item boolean \item string \item double\end{itemize}
\begin{lstlisting}[caption=Aufgabe 1.32]
private String name;
\end{lstlisting}
\begin{lstlisting}[caption=Aufgabe 1.33]
public void send(String parameter)
\end{lstlisting}
\begin{lstlisting}[caption=Aufgabe 1.34]
public int average(int a, int b)
\end{lstlisting}

\item $~$
\end{enumerate}

\subsection{Auftrag Lernteam}

\begin{lstlisting}[caption=Test von lstlisting]
/**
* Return the area of the circle
*/
public float getArea()
{
    return (float) (diameter * diameter * Math.PI / 4.0);
}
\end{lstlisting}