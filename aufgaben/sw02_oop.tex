% coding:utf-8

%PRG1, a LaTeX-Code for a the solution of the exercises from "programming and algorithms"
%Copyright (C) 2013, Daniel Winz

%This program is free software; you can redistribute it and/or
%modify it under the terms of the GNU General Public License
%as published by the Free Software Foundation; either version 2
%of the License, or (at your option) any later version.

%This program is distributed in the hope that it will be useful,
%but WITHOUT ANY WARRANTY; without even the implied warranty of
%MERCHANTABILITY or FITNESS FOR A PARTICULAR PURPOSE.  See the
%GNU General Public License for more details.
%----------------------------------------

\section{SW2}
\subsection{Selfstudy-Questions OOP2}
\subsubsection*{Exercise 4}
\textit{A class is build by three essential components. What are they?}\\

\subsubsection*{Exercise 5}
\textit{What is the order of the three components?}\\

\subsubsection*{Exercise 6}
\textit{What's their purpose?}\\

\subsubsection*{Exercise 8}
\textit{What is a variable?}\\

\subsubsection*{Exercise 9}
\textit{What are the synonyms to instance variables?}\\

\subsubsection*{Exercise 10}
\textit{What do you think where the term instance variable comes from?}\\

\subsubsection*{Exercise 11}
\textit{How can you put comments into a Java-Code?}\\

\subsubsection*{Exercise 12 (important)}
\textit{With which access-modification do you declare instance variables
	usually? Is it \lstinline{private} or \lstinline{public}? Do you
	have a reason for your answer?}\\

\subsubsection*{Exercise 13}
\textit{Explain the relation between a constructor and the state of an 
	onject.}\\

\subsubsection*{Exercise 14}
\textit{How do we name constructors?}\\

\subsubsection*{Exercise 15}
\textit{How long are the variables of an object alive (reachable)?}\\

\subsubsection*{Exercise 16}
\textit{Why sould you (if possible) initialise instance variables explicit?}\\

\subsubsection*{Exercise 17}
\textit{What's the defualt value which is given to a \lstinline{int} variavle
	by its initialisation?}\\

\subsubsection*{Exercise 19}
\textit{What's the use of parameters?}\\

\subsubsection*{Exercise 20}
\textit{What's the difference between a formal and a actual parameter?}\\

\subsubsection*{Exercise 21}
\textit{Is the following statement correct; "formal parameters are special
	variables"?}\\

\subsubsection*{Exercise 22}
\textit{What's about the accessability of formal parameters?}\\

\subsubsection*{Exercise 23}
\textit{In which way this differs from instance variables?}\\

\subsubsection*{Exercise 24}
\textit{How do the lifecycles of formal parameters and instance variables 
	differ?}\\

\subsubsection*{Exercise 26}
\textit{How would you translate the expressions "assignment" and 
	"expression"?}\\

\subsubsection*{Exercise 27}
\textit{How does an assignment-instruction work exactly? What's about to 
	be aware of in relation to data types?}\\


