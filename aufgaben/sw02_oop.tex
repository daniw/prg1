% coding:utf-8

%PRG1, a LaTeX-Code for a the solution of the exercises from "programming and algorithms"
%Copyright (C) 2013, Daniel Winz

%This program is free software; you can redistribute it and/or
%modify it under the terms of the GNU General Public License
%as published by the Free Software Foundation; either version 2
%of the License, or (at your option) any later version.

%This program is distributed in the hope that it will be useful,
%but WITHOUT ANY WARRANTY; without even the implied warranty of
%MERCHANTABILITY or FITNESS FOR A PARTICULAR PURPOSE.  See the
%GNU General Public License for more details.
%----------------------------------------

\section{SW2}
\subsection{Selfstudy-Questions OOP2}
\subsubsection*{Exercise 4}
\textit{A class is build by three essential components. What are they?}\\
fields, constructor, methods

\subsubsection*{Exercise 5}
\textit{What is the order of the three components?}\\
fields \\
constructor \\
methods

\subsubsection*{Exercise 6}
\textit{What's their purpose?}\\
fields: static variables \\
constructor: initial setting up of an object \\
methods:interaction with the objects of this class

\subsubsection*{Exercise 8}
\textit{What is a variable?}\\
A memory cell in which a value can be stored. 

\subsubsection*{Exercise 9}
\textit{What are the synonyms to instance variables?}\\
fields

\subsubsection*{Exercise 10}
\textit{What do you think where the term instance variable comes from?}\\


\subsubsection*{Exercise 11}
\textit{How can you put comments into a Java-Code?}\\
\lstinline{/* Block comment */}
// Line comment?

\subsubsection*{Exercise 12 (important)}
\textit{With which access-modification do you declare instance variables
	usually? Is it \lstinline{private} or \lstinline{public}? Do you
	have a reason for your answer?}\\
\lstinline{private} because it is not reachable from outside the object. 

\subsubsection*{Exercise 13}
\textit{Explain the relation between a constructor and the state of an 
	object.}\\
The constructor creates the initial state of an object. 

\subsubsection*{Exercise 14}
\textit{How do we name constructors?}\\
initialization

\subsubsection*{Exercise 15}
\textit{How long are the variables of an object alive (reachable)?}\\
As long as the object is alive. 

\subsubsection*{Exercise 16}
\textit{Why sould you (if possible) initialise instance variables explicit?}\\
To not rely on the default initialisation

\subsubsection*{Exercise 17}
\textit{What's the defualt value which is given to a \lstinline{int} variavle
	by its initialisation?}\\
0

\subsubsection*{Exercise 19}
\textit{What's the use of parameters?}\\
to call a method with a variable

\subsubsection*{Exercise 20}
\textit{What's the difference between a formal and a actual parameter?}\\
a formal parameter is the name of a parameter whereas a actual parameter is its 
value. 

\subsubsection*{Exercise 21}
\textit{Is the following statement correct; "formal parameters are special
	variables"?}\\
yes

\subsubsection*{Exercise 22}
\textit{What's about the accessability of formal parameters?}\\
formal parameters are accessible only from inside the method. 

\subsubsection*{Exercise 23}
\textit{In which way this differs from instance variables?}\\
instance variables are accessible from inside the class. 
\subsubsection*{Exercise 24}
\textit{How do the lifecycles of formal parameters and instance variables 
	differ?}\\
actual parameters are deleted after the method has finished. 

\subsubsection*{Exercise 26}
\textit{How would you translate the expressions "'assignment"' and 
	"'expression"'?}\\
Zuweisung \\
Ausdruck

\subsubsection*{Exercise 27}
\textit{How does an assignment-instruction work exactly? What's about to 
	be aware of in relation to data types?}\\
the value from the right side of the \lstinline{=} is copied to the left side of it. 
\\
both need to be of the same data type. 

\subsection{Exercises from book (chapter 2)}
\subsubsection*{Exercise 2.1}
\textit{Create a \lstinline{TicketMachine} object bench and take a look at its 
methods. You should see the following: 
\lstinline{get Balance, getPrice, insertMoney,} and \lstinline{printTicket}. Try out the 
\lstinline{getPrice} method. You should see a return value containing the price of 
the ticket thet was set when this object was created. Use the 
\lstinline{insertMoney} method to simulate inserting an amount of money into the 
machine. The machine stores as a balance the amount of money inserted. Use 
\lstinline{getBalance} to check that the machine has kept an accurate record of the 
amount just inserted. You can insert several seperate amounts of money into the 
machine, just like you might insert multiple coins or bills into a real 
machine. Try inserting the exact amount required for a ticket, and use 
\lstinline{getBalance} to ensure that the balance is increased correctly. As this 
is a simple machine, a ticket will not be issued automatically, so once you 
have inserted enough money, call the \lstinline{printTicket} method. A facsimile 
ticket should be printed in the BlueJ terminal window. }

\subsubsection*{Exercise 2.2}
\textit{What value is returned if you get the machine's balance after it has 
printed a ticket? }

\subsubsection*{Exercise 2.3}
\textit{Experiment with inserting different amounts of money before printing 
tickets. Do you notice anything strange about the machine's behavior? what 
happens if you insert too much money into the machine - do you receive any 
refund? what happens if you do not insert enough and then try to print a 
ticket? }

\subsubsection*{Exercise 2.4}
\textit{Try to obtain a good understanding of the ticket machine's behavior by 
interacting with it on the object bench before we start looking, in the next 
section, at how the \lstinline{TicketMachine} class is implemented. }

\subsubsection*{Exercise 2.5}
\textit{Create another ticket machine for tickets of a different price; 
remember that you have to supply this value when you create the machine object. 
Buy a ticket from that machine. Does the printed ticket look any different 
from those printed by the first machine? }

\subsubsection*{Exercise 2.6}
\textit{Write out what you think the outer wrappers of the \lstinline{Student} and 
\lstinline{LabClass} classes look like; do not worry about the inner part. }

\subsubsection*{Exercise 2.7}
\textit{Does it matter whether we write \lstinline{public class TicketMachine} or 
\lstinline{class public TicketMachine} in the outer wrapper of the class? Edit the 
source of the \lstinline{TicketMachine} class to make the change, and then close 
the editor window. Do you notice a change in the class diagram? }

\subsubsection*{Exercise 2.8}
\textit{Check whether or not it is possible to leave out the word \lstinline{public} 
from the outer wrapper of the \lstinline{TicketMachine} class. }

\subsubsection*{Exercise 2.9}
\textit{Put back the word \lstinline{public}, and then check whether it is possible 
to leave out the word \lstinline{class} by trying to compile again. Make sure that 
both words are put back as they were originally before continuing. }

\subsubsection*{Exercise 2.10}
\textit{From your earlier experimentation with the ticket machine objects 
within BlueJ, you can probably remember the names of some of the methods-
\lstinline{printTicket}, for instance. Look at the class definition in Code 2.1, 
and use this knowledge, along with the additional information about ordering 
we have given you, to make a list of the names of the fields, constructors and 
methods in the \lstinline{TicketMachine} class. Hint: There is only one constructor 
in the class. }

\subsubsection*{Exercise 2.11}
\textit{What are the two features of the constructor that make it look 
signifigantly different from the methods of the class? }

\subsubsection*{Exercise 2.12}
\textit{What do you think is the type of each of the following fields? \\
\lstinline{private int count;}\\
\lstinline{private Student representative;}\\
\lstinline{private Game game;}}

\subsubsection*{Exercise 2.13}
\textit{What are the names of the following fields? \\
\lstinline{private boolean alive;}\\
\lstinline{private Person tutor;}\\
\lstinline{private Game game;}}

\subsubsection*{Exercise 2.14}
\textit{From what you know about the naming conventions for classes, which of 
the type names in Exercises 2.12 and 2.13 would you say are class names? }

\subsubsection*{Exercise 2.15}
\textit{In the following fiels declaration from the \lstinline{TicketMachine} class\\
\lstinline{private int price;}\\
does it matter which order the three words appear in? Edit the 
\lstinline{TicketMachine} class to try different orderings. After each change, 
close the editor. Does the appearance of the class diagram after each change 
give you a clue as to whether or not other orderings are possible? Check by 
pressing the Compile button to see if there is an error message. }

\subsubsection*{Exercise 2.16}
\textit{Is it always necessary to have a semicolon at the end of a field 
declaration? Once again, expreiment via the editor. The rule you will learn 
here is an important one, so be sure to remember it. }

\subsubsection*{Exercise 2.17}
\textit{Write in full the declaration for a field of type \lstinline{int} whose 
name is \lstinline{status}. }

\subsubsection*{Exercise 2.18}
\textit{To what class does the following constructor belong? \\
\lstinline{public Student(String name)}}

\subsubsection*{Exercise 2.19}
\textit{How many parameters does the following constructor have, and what are 
their types? \\
\lstinline{public Book(String title, double, price)}}

\subsubsection*{Exercise 2.20}
\textit{Can you guess what types some of the \lstinline{Book} class's fields might 
be, from the parameters in its constructor? Can you assume anything about the 
names of the fields? }

\subsubsection*{Exercise 2.21}
\textit{Suppose that the class \lstinline{Pet} has a field called \lstinline{name} that 
is of type \lstinline{String}. Write an assignment in the body of the following 
constructor so that the \lstinline{name} field will be initialized with the value 
of the constructor's parameter. \\
\lstinline{public Pet(String petsName)}\\
\lstinline!\{!\\
\lstinline!\}!}

\subsubsection*{Exercise 2.22}
\textit{Challange exercise The following object creation will result in the 
constructor of the \lstinline{Date} class being called. Can you write the 
constructor's header? \\
\lstinline{new Date("March", 23, 1861)}\\
Try give meaningful names to the parameters. }

