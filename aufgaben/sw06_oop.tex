\section{SW6}
\subsection{Selfstudy-Questions OOP6}

\subsubsection{Chapter 5.2 - The TechSupport system}

\subsubsection*{Exercise 1}
\textit{Solve the exercise 5.1}

\subsubsection*{Exercise 2}
\textit{At page 157 and 158 there is a Method start() that uses a 
while-loop. Create a code snippet with the same functionality by using a
do-while loop.}
\begin{lstlisting}
public void start() {
    boolean finished = false;
    
    printWelcome();
    do {
        String input = reader.getInput();
        
        if(input.startsWith("bye")) {
            finished = true;
        }
        else {
            String response = responder.generateResponse();
            System.out(response);
        }
    } while(!finished);
    printGoodbye();
}
\end{lstlisting}

\subsubsection{Chapter 5.3 - Reading class documentation}

\subsubsection*{Exercise 3}
\textit{Solve the exercises 5.2 to 5.5 as well as 5.7 to 5.11}

\subsubsection*{Exercise 4}
\textit{In which package do you suppose the class FileWriter?
Check your guess with the Java API documentation.}
io

\subsubsection*{Exercise 5}
\textit{With the class BufferReader you can read files line by line.
How does that work? You can find the answer in the Java API documentation.}

\subsubsection{Chapter 5.4 - Adding random behavior}

\subsubsection*{Exercise 6}
\textit{Solve the exercises 5.12 and 5.13}

\subsubsection*{Exercise 7}
\textit{Solve the exercise 5.15}

\subsubsection*{Exercise 8}
\textit{Solve the exercise 5.18}

\subsubsection{Chapter 5.5 - Packages and import}

\subsubsection*{Exercise 9}
\textit{Solve the exercises 5.21 and 5.22}

\subsection{Questions from the book}

\subsubsection{5.1}

\subsubsection{5.2}

\subsubsection{5.3}
Both methods tests whether the string starts with the prefix given. The method 
with two parameters allows to define where the start of the string should be. 

\subsubsection{5.4}
\verb{endsWith()} \\
parameter: String suffix \\
return type: boolean

\subsubsection{5.5}
\verb{length()} \\
no parameters, \\
return type: int

\subsubsection{5.7}

\subsubsection{5.8}

\subsubsection{5.9}

\subsubsection{5.10}

\subsubsection{5.11}

\subsubsection{5.12}

\subsubsection{5.13}

\subsubsection{5.15}

\subsubsection{5.18}

\subsubsection{5.21}

\subsubsection{5.22}

